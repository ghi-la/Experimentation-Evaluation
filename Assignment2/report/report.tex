\documentclass{article}
\usepackage[utf8]{inputenc}
\usepackage{amsmath}
\usepackage{graphicx}
\usepackage{amssymb}
\usepackage{subcaption}
\usepackage{placeins}
\usepackage{url}
\usepackage[a4paper, margin=0.8in]{geometry}

\title{What is the best performing sorting algorithm? \\ \large Experimental analysis commissioned by Bubble Inc}
\author{Ghilardini Matteo, Toscano Sasha}
\date{\today}

\begin{document}

\maketitle

\section*{Abstract}



\section{Introduction}


\subsection*{Hypothesis}


\newpage

\section{Method}
\subsection{Variables}
\begin{table}[h!]
    \centering
    \begin{tabular}{l|c|l}
        \textbf{Independent variable} & \textbf{Levels} & \textbf{Description}                               \\ \hline
        Different algorithms          & 4               & 4 different algorithms                             \\
        Data types                    & 3               & Integer (4 bytes), Char (2 bytes), Boolean (1 bit) \\
        Size of the array             & 2               & Integer number, we selected 1,000 and 10,000       \\
        Starting array order          & 3               & Sorted, Shuffled, or Reverse-sorted                \\
    \end{tabular}
    \caption{Independent variables}
\end{table}

\begin{table}[h!]
    \centering
    \begin{tabular}{l|c|l}
        \textbf{Dependent variable}  & \textbf{Measurement scale} & \textbf{Measurement method} \\
        \hline
        Time to complete the sorting & nanoseconds                & \texttt{System.nanoTime()}  \\
    \end{tabular}
    \caption{Dependent variables}
\end{table}

\begin{table}[h!]
    \centering
    \begin{tabular}{l|l}
        \textbf{Control variable}     & \textbf{Fixed value}                                \\ \hline
        Number of active applications & None outside of IDE and WSL Terminal                \\
        Computer HW, OS, JDK          & Intel i7-9700k, 32GB RAM, Windows 10, OpenJDK 23.01 \\
    \end{tabular}
    \caption{Control variables}
\end{table}

\FloatBarrier

\subsection{Design}
\subsubsection{Study Type}

\begin{table}[!h]
    \centering
    \begin{tabular}{c|c|c}
        \textbf{Observational Study} & \textbf{Quasi-Experiment} & \textbf{Experiment} \\ \hline
        \(\square\)                  & \(\square\)               & \(\boxtimes\)       \\
    \end{tabular}
    \caption{Study type selection}
\end{table}



\subsubsection{Number of Factors}
\begin{table}[!h]
    \centering
    \begin{tabular}{c|c|c}
        \textbf{Single-Factor Design} & \textbf{Multi-Factor Design} & \textbf{Other} \\ \hline
        \(\square\)                   & \(\boxtimes\)                & \(\square\)    \\
    \end{tabular}
    \caption{Number of factors selection}
\end{table}

\subsection{Participants}

\subsection{Apparatus and Materials}


\subsection{Procedure}

\newpage

\section{Results}
\subsection{Visual Overview}


\subsection{Descriptive Statistics}


\subsection{Inferential Statistics}

\clearpage
\newpage

\section{Discussion}
\subsection{Compare Hypothesis to Results}


\subsection{Limitations and Threats to Validity}


\subsection{Conclusions}

\newpage

\section{Appendix}\label{chap:appendix}
\subsection{Materials}

\end{document}